\documentclass[12pt,oneside,openright,final,brazil,a4paper,top=3.5cm,left=3cm,
right=3cm,bottom=2.5cm]{memoir}
\usepackage[brazilian]{babel}
\usepackage[latin1]{inputenc}
\usepackage{setspace}
\usepackage{graphicx}

\setlrmarginsandblock{3cm}{2cm}{*}
\setulmarginsandblock{3cm}{3cm}{*}
%\setheadfoot{1cm}{1cm}
\setheaderspaces{2cm}{*}{*}

\checkandfixthelayout

\makeheadrule{myheadings}{\textwidth}{\normalrulethickness}
\makeoddhead{myheadings}{\textsc{\leftmark}}{}{\thepage}    %
\makeevenhead{myheadings}{\thepage}{}{\textsc{\leftmark}}

%\copypagestyle{contents}{myheadings}
%\makeoddhead{contents}{\textsc{\contentsname}}{}{\thepage}
%\makeevenhead{contents}{\thepage}{}{\textsc{\contentsname}}

\copypagestyle{bibliography}{myheadings}
\makeoddhead{bibliography}{\textsc{\bibname}}{}{\thepage}
\makeevenhead{bibliography}{\thepage}{}{\textsc{\bibname}}


\makeatletter

\def\mychaptermark#1{%
    \markboth{%
    \ifnum \c@secnumdepth >\m@ne
        \if@mainmatter
%           \@chapapp\ \thechapter. \ %
            \thechapter. \ %
        \fi
    \fi
    #1}{}}%

\def\mysectionmark#1{%
\markright{%
    \ifnum \c@secnumdepth > \z@
        \thesection. \ %
    \fi
    #1}}%

\makeatother

\maxsecnumdepth{subsubsection}


\sloppy

\newcommand{\tableformat}{\small\centering}
\newcommand{\figureformat}{\centering}

\newcommand{\tittcc}{
    \textsf{\bfseries\LARGE Modelo de TCC do Curso de Ci�ncia da\\
    \vspace{0.8ex}
    Computa��o do UniNorte}
}

\newcommand{\descrtcc}{
\hspace{\stretch{1}}\parbox{0.51\textwidth}{
\textbf{Trabalho de Conclus�o de Curso} apresentado � Coordena��o de Ci�ncia da
Computa��o do Centro Universit�rio do Norte, como parte dos requisitos
necess�rios para a obten��o do t�tulo de Bacharel em Ci�ncia da Computa��o.
}}

\begin{document}

\frontmatter
\pagestyle{empty}

% -- Capa ---------------------------------------------------------------------
{\centering
\textsf{\large
CENTRO UNIVERSIT�RIO DO NORTE\\
Laureate International Universities\\
\vspace*{\stretch{0.03}}
CURSO DE CI�NCIA DA COMPUTA��O
}

\vspace*{\stretch{0.2}}

{\large NOME DO AUTOR 1\\}
{\large NOME DO AUTOR 2\\}
{\large NOME DO AUTOR 3\\}

\vspace*{\stretch{0.3}}

\tittcc

\vspace*{\stretch{1}}

MANAUS \\
2010

}
\cleardoublepage

% -- Folha de rosto -----------------------------------------------------------
{\centering
{\large Nome do Autor}

\vspace*{\stretch{2}}

\tittcc

\vspace*{\stretch{1}}

\descrtcc

\vspace*{\stretch{1}}

Orientador: Nome do Orientador

}
\cleardoublepage

% -- Ficha Catalogr�fica ---------------------------------------------------
\begin{center}
{\Large Ficha Catalogr�fica}
\cleardoublepage
\end{center}

% -- Banca Examinadora --------------------------------------------------------
{\centering

{\large Nome do Autor}

\vspace*{\stretch{2}}

\tittcc

\vspace*{\stretch{1}}

\descrtcc

\vspace*{\stretch{1}}

\begin{center}
\large Banca Examinadora
\end{center}

\vspace{1em}

Prof. Nome do Professor(a) -- Orientador\\
Institui��o do professor
\vspace{1em}

Prof. Nome do Professor(a)\\
Institui��o do professor
\vspace{1em}

Prof. Nome do Professor(a)\\
Institui��o do professor

\vspace*{\stretch{0.5}}

\begin{center}
Manaus\\
2010
\end{center}
}
\cleardoublepage

% -- Dedicat�ria --------------------------------------------------------------
\vspace*{\stretch{2}}

\hspace{\stretch{1}}\textit{Texto da dedicat�ria}\hspace{1cm}

\vspace*{\stretch{1}}

\cleardoublepage

% -- Agradecimentos -----------------------------------------------------------

\chapter*{Agradecimentos}
\thispagestyle{empty}

{\setlength{\parindent}{0cm}

Texto dos agradecimentos.
}
\cleardoublepage

% -- Ep�grafe -----------------------------------------------------------------

{\vspace*{\stretch{1}}

\hspace{\stretch{1}}\parbox{0.5\textwidth}{%
Texto da ep�grafe.
}
\vspace{1em}

\hspace{\stretch{1}}{\emph{Autor}}
}
\cleardoublepage

% -- Resumo -------------------------------------------------------------------

\chapter*{Resumo}
\thispagestyle{empty}

{Resumo do trabalho deve ser colocado aqui.

\vspace*{\stretch{1}}

\noindent \textsf{Palavras-chave:} Lista de palavras separadas por v�rgula.
}
\cleardoublepage

% -- Abstract -----------------------------------------------------------------

\chapter*{Abstract}
\thispagestyle{empty}

{Abstract's text.

\vspace*{\stretch{1}}

\noindent \textsf{Keywords:} Words
}
\cleardoublepage

% -- �ndices ------------------------------------------------------------------
\pagenumbering{roman}
\pagestyle{plain}

\tableofcontents*  % gera o �ndice (sum�rio)
\cleardoublepage

\listoffigures*    % gera a lista de figuras
\cleardoublepage

\listoftables*     % gera a lista de tabelas
\cleardoublepage

\mainmatter
\pagestyle{myheadings}
\let\chaptermark=\mychaptermark
\let\sectionmark=\mysectionmark

\chapter{Introdu��o}
\label{cap:cap_1}
Fazer a introdu��o ao trabalho.

\chapter{Cap�tulo de Desenvolvimento}
\label{cap:cap_2}
Inserir o texto aqui.

\section{Exemplo de Tabela}
\label{sec:exemplo_tabela}
Apenas um exemplo de como criar uma tabela em \LaTeX{}.

\section{Exemplo de Figura}

\section{Exemplo de Refer�ncia}
\subsection{Refer�ncia externa}
Usar refer�ncia de livros, links, etc. Exemplo: De acordo com \cite{Knuth78}...

\subsection{Refer�ncia interna}
Podemos referenciar um cap�tulo, se��o ou qualquer parte do documento
utilizando o caption. Exemplo: citando o Cap�tulo \ref{cap:cap_1} ou a se��o
\ref{sec:exemplo_tabela}. 

\chapter{Conclus�es}
\label{cap:cap_3}
Inserir o texto aqui.

\section{Exemplos de Listas}
\subsection{Lista Simples}
Criando lista simples com \textit{bullets} ou outro caractere, como por exemplo
o h�fen.
\begin{itemize}
  \item Item 1.
  \item Item 2.
  \item Item 3.
\end{itemize}

\begin{itemize}
  \item[-] Item 1.
  \item[-] Item 2.
  \item[-] Item 3.
\end{itemize}

\subsection{Lista Enumerada}
Criando lista enumerada.
\begin{enumerate}
  \item Item A.
  \item Item B.
  \item Item C. 
\end{enumerate}

\section{Exemplo de Tabela}
\label{sec:exemplo_tabela}
Apenas um exemplo de como criar uma tabela em \LaTeX{}. O exemplo pode ser
visto na tabela \ref{tab:exemplo_tabela}.

  \begin{table}[htbp]
   \centering
    \begin{tabular}{|c|c|}
      \hline
      \multicolumn{1}{|c|}{\textbf{Coluna 1}} &
      \multicolumn{1}{|c|}{\textbf{Coluna 2}}\\
      \hline
      bla bla bla bla & bla bla bla \\
      \hline
    \end{tabular}
    \caption{Exemplo de uma tabela}
    \label{tab:exemplo_tabela}
  \end{table}

\section{Exemplo de Figura}
Um simples exemplo de como inserir uma figura no documento. A figura
\ref{fig:uninorte} contempla este exemplo.

\begin{figure}
\centering
\includegraphics[width=8cm]{imagens/marca_uninorte.png}
\caption{Marca do UniNorte}
\label{fig:uninorte}
\end{figure}

\section{Exemplo de Refer�ncia}
\subsection{Refer�ncia externa}
Usar refer�ncia de livros, links, etc. Exemplo: De acordo com \cite{Knuth78}...

\subsection{Refer�ncia interna}
Podemos referenciar um cap�tulo, se��o ou qualquer parte do documento
utilizando o caption. Exemplo: citando o Cap�tulo \ref{cap:cap_1} ou a se��o
\ref{sec:exemplo_tabela}. 

\section{Exemplos Matem�ticos}
Alguns exemplos de como escrever exemplos matem�ticos.

\begin{enumerate}
  \item �rea de um c�rculo:
    \begin{center}
      \begin{displaymath}
        a = \pi R^{2}
      \end{displaymath}
    \end{center}
  \item F�rmula de Bhaskara:
    \begin{center}
      \begin{displaymath}
        x = \frac{-b \pm \sqrt{b^{2} -4ac}}{2a}
      \end{displaymath}
    \end{center}
  \item Limite:
    \begin{center}
      \begin{displaymath}
        \lim_{x\rightarrow a} f(x) = A
      \end{displaymath}
    \end{center}
  \item Derivada:
    \begin{center}
      \begin{displaymath}
        \frac{dy}{dx}
      \end{displaymath}
    \end{center}
  \item Integral:
    \begin{center}
      \begin{displaymath}
        \int(x^{3} + 5x^{2} + 3x - 5)dx
      \end{displaymath}
    \end{center}
  \item Somat�rio:
    \begin{center}
      \begin{displaymath}
        \sum_{i=1}^{10} i + 3
      \end{displaymath}
    \end{center}
  \item Matriz:
    \begin{center}
      \begin{displaymath}
        \mathbf{M} = 
          \left[ \begin{array}{ccc}
          m_{11} & m_{12} & \ldots \\
          m_{21} & m_{22} & \ldots \\
          \vdots & \vdots & \ddots
          \end{array} \right]
      \end{displaymath}
    \end{center}

\end{enumerate}


\cleardoublepage
\renewcommand{\bibname}{Refer�ncias}
\pagestyle{bibliography}
\bibliographystyle{apalike}
\bibliography{uninorte-template}
\end{document}
