Inserir o texto aqui.

\section{Exemplos de Listas}
\subsection{Lista Simples}
Criando lista simples com \textit{bullets} ou outro caractere, como por exemplo
o h�fen.
\begin{itemize}
  \item Item 1.
  \item Item 2.
  \item Item 3.
\end{itemize}

\begin{itemize}
  \item[-] Item 1.
  \item[-] Item 2.
  \item[-] Item 3.
\end{itemize}

\subsection{Lista Enumerada}
Criando lista enumerada.
\begin{enumerate}
  \item Item A.
  \item Item B.
  \item Item C. 
\end{enumerate}

\section{Exemplo de Tabela}
\label{sec:exemplo_tabela}
Apenas um exemplo de como criar uma tabela em \LaTeX{}. O exemplo pode ser
visto na tabela \ref{tab:exemplo_tabela}.

  \begin{table}[htbp]
   \centering
    \begin{tabular}{|c|c|}
      \hline
      \multicolumn{1}{|c|}{\textbf{Coluna 1}} &
      \multicolumn{1}{|c|}{\textbf{Coluna 2}}\\
      \hline
      bla bla bla bla & bla bla bla \\
      \hline
    \end{tabular}
    \caption{Exemplo de uma tabela}
    \label{tab:exemplo_tabela}
  \end{table}

\section{Exemplo de Figura}
Um simples exemplo de como inserir uma figura no documento. A figura
\ref{fig:uninorte} contempla este exemplo.

\begin{figure}
\centering
\includegraphics[width=8cm]{imagens/marca_uninorte.png}
\caption{Marca do UniNorte}
\label{fig:uninorte}
\end{figure}

\section{Exemplo de Refer�ncia}
\subsection{Refer�ncia externa}
Usar refer�ncia de livros, links, etc. Exemplo: De acordo com \cite{Knuth78}...

\subsection{Refer�ncia interna}
Podemos referenciar um cap�tulo, se��o ou qualquer parte do documento
utilizando o caption. Exemplo: citando o Cap�tulo \ref{cap:cap_1} ou a se��o
\ref{sec:exemplo_tabela}. 

\section{Exemplos Matem�ticos}
Alguns exemplos de como escrever exemplos matem�ticos.

\begin{enumerate}
  \item �rea de um c�rculo:
    \begin{center}
      \begin{displaymath}
        a = \pi R^{2}
      \end{displaymath}
    \end{center}
  \item F�rmula de Bhaskara:
    \begin{center}
      \begin{displaymath}
        x = \frac{-b \pm \sqrt{b^{2} -4ac}}{2a}
      \end{displaymath}
    \end{center}
  \item Limite:
    \begin{center}
      \begin{displaymath}
        \lim_{x\rightarrow a} f(x) = A
      \end{displaymath}
    \end{center}
  \item Derivada:
    \begin{center}
      \begin{displaymath}
        \frac{dy}{dx}
      \end{displaymath}
    \end{center}
  \item Integral:
    \begin{center}
      \begin{displaymath}
        \int(x^{3} + 5x^{2} + 3x - 5)dx
      \end{displaymath}
    \end{center}
  \item Somat�rio:
    \begin{center}
      \begin{displaymath}
        \sum_{i=1}^{10} i + 3
      \end{displaymath}
    \end{center}
  \item Matriz:
    \begin{center}
      \begin{displaymath}
        \mathbf{M} = 
          \left[ \begin{array}{ccc}
          m_{11} & m_{12} & \ldots \\
          m_{21} & m_{22} & \ldots \\
          \vdots & \vdots & \ddots
          \end{array} \right]
      \end{displaymath}
    \end{center}

\end{enumerate}
